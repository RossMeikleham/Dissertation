\section{Wool}
Wool\cite{wool} is a library written for C that supports task parallel programming. It is inspired by Cilk and introduces
three forms of synchronization (Create, Exit, and Join). Instead of being a language extension like Cilk, macros and
inline functions are used for low overhead in task operations. 

Wool relies on tasks having low overheads, and focuses on having low overhead of task creation for fine grained task parallelism. 
Tasks can only synchronize when a task is created or completes execution. To reduce overhead, if every thread
in the system is busy, further tasks are executed sequentially as procedure calls which saves overhead in having
to create actual threads.

Wool implements a work-stealing scheduler.
Each thread is given its own task dequeue. A \texttt{SPAWN} operation pushes a task onto 
one of the queues, and a \texttt{SYNC} operation removes a task from the tail of the queue.
Tasks are taken from the front of the queue and executed. Only the owner thread of the queue can remove a task from that queue. 
If a thread's task dequeue is empty it can steal a task from the front of one of the other queues.

