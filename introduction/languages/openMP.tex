\subsection{Open MP}

OpenMP (Open Multi-Processing) is a language extension available for C, C++ and Fortran which allows for shared memory multi-processing. 
This is achieved by the use of compiler directives (more specifically in C/C++ this is done through the preprocessor using pragmas) and
the OpenMP API.\cite{openmp}.

OpenMP's use of pragmas for parallel programming means that parallel code keeps sequential semantics
and any compiler which doesn't implement OpenMP extensions can still compile the code by ignoring the pragmas.
The results of executing the program serially without the pragmas should be exactly the same as when
the sections are parallelized. Another benefit is that sections of serial programs can be
parallelized by adding pragmas and existing code doesn't need to be modified.

For example, given a simple for loop below:

\begin{lstlisting}[style=myGPC]
for(int i = 0; i < 10; i++) {
    arr[i] =  do_calculation(i);
}
\end{lstlisting}

We can parallelize this code by simply adding a pragma above
the for loop:

\begin{lstlisting}[style=myGPC] 
#pragma omp parallel for
for(int i = 0; i < 10; i++) {
    arr[i] =  do_calculation(i);
}
\end{lstlisting}

OpenMP also supports task parallelism as of version 3.0\cite{openmp3}. The specification for OpenMP
does not specify how the scheduler should work, and no specific implementations of OpenMP appear to have 
implemented a task stealing scheduler.

A downside to using pragmas is that accidentally missing certain directive keywords may cause undesirable 
program behaviour such as unecessary parallelization\cite{openmptraps}, and no warnings will be given by the compiler.

 
