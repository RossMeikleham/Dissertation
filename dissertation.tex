\pdfoutput=1

\documentclass{l4proj}
\usepackage{listings}
\usepackage{textcomp}
\usepackage{xcolor}
\usepackage{url}

\begin{document}

%Set the Font/Coloring for the GPC language,
%Use the C++ default with added seq and par keywords
\lstdefinestyle{myGPC} {language=C++,
                basicstyle=\ttfamily,
                keywordstyle=\color{blue}\ttfamily,
                stringstyle=\color{red}\ttfamily,
                commentstyle=\color{gray}\ttfamily,
                morecomment=[l][\color{magenta}]{\#}
                directivestyle={\color{green}}
                identifierstyle=\color{purple},
                numbers=left,
                morekeywords={seq}
}


\lstdefinestyle{myGPIR} {language=Lisp,
                basicstyle=\ttfamily,
                keywordstyle=\color{blue}\ttfamily,
                stringstyle=\color{red}\ttfamily,
                commentstyle=\color{gray}\ttfamily,
                morecomment=[l][\color{magenta}]{\#}
                directivestyle={\color{green}}
                identifierstyle=\color{purple},
                numbers=left,
                morekeywords={seq}
}

\lstdefinestyle{myHaskell} {language=Haskell,
                basicstyle=\ttfamily,
                keywordstyle=\color{blue}\ttfamily,
                stringstyle=\color{red}\ttfamily,
                commentstyle=\color{gray}\ttfamily,
                morecomment=[l][\color{magenta}]{\#}
                directivestyle={\color{green}}
                identifierstyle=\color{purple},
                numbers=left,
                morekeywords={seq}
}


\title{Design and Compilation of a C-like front-end language for GPRM}
\author{Ross Meikleham}
\date{March 2015}
\maketitle


\begin{abstract}
\begin{abstract}
In the past most software was written with serial computation in mind 
and increases in processor speeds over time would in turn increase the speed 
the software ran at.
However with the rate of increase in processor speeds declining and 
manufacturers focusing on adding more processor cores this is no longer the case.
The result is that serially written software needs to be rewritten
if it wishes to take advantage of multiple processors. This project
focuses on designing a language which is familiar to most programmers
that describes the composition of serial tasks written in C++ and
can be executed in parallel. 
\end{abstract}

\end{abstract}

\educationalconsent
%
%NOTE: if you include the educationalconsent (above) and your project is graded an A then
%      it may be entered in the CS Hall of Fame
%
\tableofcontents
%==============================================================================

\pagenumbering{arabic}
\setlength{\parindent}{0pt}

\chapter{Introduction}

\pagenumbering{arabic}
\section{GPRM}

\subsection{What is the GPRM}

The Glasgow Parallel Reduction Machine is a virtual machine framework for multi-core programming using a task-based approach. It allows the programmer to structure their programs as a seperation of task-code (written as C++ classes) and communication code. 

Communication code is currently written in a language called GPIR (Glasgow Parallel Intermediate Representation). 
GPIR code controls how tasks communicate with one another and whether groups
of tasks can be evaluated sequentially or in parallel.  GPIR code is compiled down further to GPRM byte-code which is evaluated by the GPRM virtual machine.

The GPRM uses task nodes which consists of a task kernel and a task manager.

Task code is represeted as a task kernel. A task kernel is a self contained unit, typically represented as a C++ class.
To create a task kernel, the C++ class needs to be in the \textit{GPRM::Kernel::namespace}.

Communication code is represented as a task manager. A task manager "co-ordinates" communication between one or more task kernels, and
is represented as a function which can be called from a C++ program.\cite{GPRM}

\newpage

\begin{figure}[ht]
\pdfimageresolution=110
\begin{center}
\includegraphics{graphs/gprm.png}
\caption{A simple overview of the GPRM framework}
\end{center}
\end{figure}

\subsection{The GPIR language}

The GPIR language is a purely functional S-expression based language that is evaluated in parallel by default 
with optional sequential evaluation semantics.

Tasks in GPIR are postifxed with a thread number to indicate to the GPRM runtime which thread
the task should run on. For example a simple GPIR program which adds numbers in parallel:

\begin{lstlisting}[style=myGPC]
(begin
    +[0] (+[0] '3 '2) (+[1] '4 '10)
)
\end{lstlisting}

The two nested additions are performed in parallel, with the first being mapped onto thread 0,
and the second being mapped onto thread 1. When they've both been evaluated, then the outer addition
with add the results on thread 0.

\subsubsection{Quoting}

Like in Scheme and Lisp, quoting deffers the evaluation. This is useful for performing sequential evaluation.

\begin{lstlisting}[style=myGPC]
(seq 
    '(obj.m1[0] '1)
    '(obj.m2[0] '2)
)
\end{lstlisting}

Due to the parallel evaluation of GPIR, if the expressions in the seq block wern't quoted then
they would get evaluated in parallel instead of being deffered to be evaluated by the seq function.

Also literal values don't need to be evaluated, they should be deferred and passed to tasks which is
the reason the numbers are quoted in these examples.

The GPIR language has other keywords and features but these are the ones that are important for
understanding the examples and design choices made for this project.

The aim of the project is to design and create a C-like front end programming language for the GPRM. The compiler should generate the GPIR code to be used with the GPRM.



\section{Current C/C++ Parallel Programming Models}

By researching available C/C++ parallel programming frameworks/language extensions 
we can determine possible features and design choices that may suitable for the GPC language.

Cilk Plus is a general purpose programming language based on C++. It extends the C++ language with features such as
parallel for loops and spawning functions in parallel using a "fork-join" model to achieve task-parallelism.

One of the main princeples of the Cilk language is that the programmer should be responsible for exposing the parallelism
in their code. The runtime should then have the responsibility of scheduling the threads, and dividing work between
processors. 



\subsection{Open MP}

OpenMP (Open Multi-Processing) is a language extension available for C, C++ and Fortran which allows for shared memory multi-processing. 
This is achieved by the use of compiler directives (more specifically in C/C++ this is done through the preprocessor using pragmas) and
the OpenMP API.\cite{openmp}.

OpenMP's use of pragmas for parallel programming means that parallel code keeps sequential semantics
and any compiler which doesn't implement OpenMP extensions can still compile the code by ignoring the pragmas.
The results of executing the program serially without the pragmas should be exactly the same as when
the sections are parallelized. Another benefit is that sections of serial programs can be
parallelized by adding pragmas and existing code doesn't need to be modified.

For example, given a simple for loop below:

\begin{lstlisting}[style=myGPC]
for(int i = 0; i < 10; i++) {
    arr[i] =  do_calculation(i);
}
\end{lstlisting}

The above code can be parallelized by simply adding a pragma above
the for loop:

\begin{lstlisting}[style=myGPC] 
#pragma omp parallel for
for(int i = 0; i < 10; i++) {
    arr[i] =  do_calculation(i);
}
\end{lstlisting}

OpenMP also supports task parallelism as of version 3.0\cite{openmp3}. The specification for OpenMP
does not specify how the scheduler should work, and no specific implementations of OpenMP appear to have 
implemented a task stealing scheduler.

A downside to using pragmas is that accidentally missing certain directive keywords may cause undesirable 
program behaviour such as unnecessary parallelization\cite{openmptraps}, and no warnings will be given by the compiler.

 

\subsection{Intel TBB}

Intell TBB (Intell Thread Building Blocks) is a portable C++ template library for task parallelism.
It contains a range of concurrent algorithms, containers, and it's own task scheduler to achieve this\cite{tbb}.

Operations are treated as tasks, and the task scheduler has the job of dynamically allocating these tasks 
to individual cores which abstracts the specific details of allocating threads from the Programmer. Like CilkPlus,
Intell TBB's task scheduler also implements task stealing for dynamic load balancing\cite{tbbschedule}.

Intell TBB relies on generic programming which allows for writing parallel algorithms based on 
requirements on types. Parallel code can be written once and can work with lots of different
types without having to rewrite the algorithm for each type \cite{tbbbenefits}.




The goal is to create a C-like language which then compiles down to GPIR code.
The language should be as C-like as possible ideally we would like it to be as much
of a complete subset of C++ as possible, as the reset of the framework (Task Kernels and calling code)
are written in C++ and is more consistent with how the entire framework works.

C++ is a statically typed, imperative language which is sequentially evaluated by default.
GPIR is a dynamically typed functional language which is evaluated in parallel by default.
Both of these are pretty much complete opposite so the language design has to be one of that
is like C++ and can be mapped onto GPIR without too much trouble.

Design decisions:

    Parallel Evaluation:
        GPIR is parallel evaluated by default, with a "seq" function which
        evaluates the given arguments in parallel. Making GPC abide by equivalent rules of parallel
        evaluation by default with optional sequential makes the most sense, and gives the
        power to do stuff. 

    Purely Functional:
        
        To allow for mapping tasks to different cores efficiently
        at compile time, the execution path must be able
        to be known at compile time. 
        
        For example given a tree-like
        recursion, 
        
        Single variable assignment
        and not allowing return values from kernel functions
        to be used in condition statements allow for this.

        To achieve this results from kernel functions are implicitly
        "tainted" as a kernel type.

        For example:
            int x = kernelObj.m1();
        has a return type of int.
        But the compiler implicitly casts this to a 
        kernel int.

        Kernel types can be passed to kernel methods, and even
        mixed with non kernel types when passed as arguments to
        kernel methods. For example:

        int y = 5;
        seq {
            int x = kernelObj.m1();
            kernelObj.m2(x + y);
        }

        would be allowed, but the following:

        int y = kernelObj.m1();



Language Features:

    This section explains the features of the language with respect to the design decisions above.

    Syntax:
        The syntax aims to be as close to C/C++ as possible. Statements must end with a semicolon. 
        All variables must be statically typed. Blocks are surrounded in braces. Case is dependent.

    New Operators:
        Two new operators not currently present in C++ "seq" and "par" are introduced. 
        These are placed before a block
        of statements to determine whether each individual statement within the block are
        to be evaluated in sequential or parallel. By default a block of statements are evaluated
        in parallel, but the par keyword makes it more explicit especially if there's lots of nested
        seq/par blocks.

    Types:
        C++ types such as string, char, bool, int, and double are included.
        Pointers, and multi-level pointer types are included.
        However pointers are restricted in that you cannot take
        an address of any variable, adding and subtracting integers
        from pointers is allowed. Usually pointers are passed into
        the GPRM from the C++ caller to represent an Array.
        
    Operations:
        Most basic binary arithmetic operations are included i.e. 
        (+, -, *, /, \%, ==, !=, &&, ||, <<, >>, &, |, ^) and
        unary operations :
        (-, ~, !) 
        are included. 

        (+=, ++, --, -=) are not included, due to the single assignment rule.
        Although an exception is made for the "afterthought" of the for loop construct, in which
        the integer loop variable can be incremented with += or decremented by -=.

    Functions:
        Exactly the same as C/C++ 


    Single assignment:
       Variables in GPC can only be assigned once per scope.

       for example the following wouldn't be allowed
           int i = 0;
           int i = i + 1;

       but the following would be:
           int i = 0;
           {
               int i = i;
           }

    For-Loops:
       For loops have the same syntax as C/C++ for loops with some restrictions.
       Essentially these are unrolled at compile-time and every iteration is evaluated in parallel.
       
       Essentialy the syntax is for(int var = n; [Boolean expression]; var += n or var -= n) 

\chapter{Implementation}


\section{Language Choice}

The compiler will be implemented using the Haskell programming language.

Functional languages are suited for building compilers.
This is due to the traversal of trees which occurs often during complication.
Functional languages typically have features such as pattern matching and tail-end recursive optimisation 
which make them suitable for traversing trees.

The Glasgow Haskell Compiler is available for most platforms, most importantly for Windows, OSX and Linux 
on x86 architectures. This allows the compiler to be portable across these platforms provided the libraries 
used to build the compiler are portable.

Haskell has algebaric datatypes, these make it easy to represent Abstract Syntax Trees in a 
readable form, and express recursive relations.

\begin{lstlisting}[style=myHaskell]

    data Expression =
          Add Expression Expression
        | Sub Expression Expression
        | Const Integer

\end{lstlisting}



Haskell also has powerful libraries for parsing source code such as Parsec which is a parser combinator
library. Parsec allows Combinator parsers to be written in the Haskell language itself avoiding the complexity
of integration of different tools and languages. 
\cite{parsec}

\section{Tools and Build Process}

\subsection{Cabal}
\subsection{Testing}
\subsection{Continuous Integration}
\subsection{Code Coverage}


Splitting up compilation into multiple stages:
    
    Parsing/Lexing into AST

         |

     Type and Scope Checking the AST

         |

     Strip annotations and types from the AST

        |

     Evaluator which steps through the code and generates a 
     GPIR AST

        |

      Translate GPIR AST into GPIR source code

\section{Parser/Lexer}
\section{Type and Scope checking}
\section{Evaluation}
\section{GPIR Code Generation}

\chapter{Future Work}

Future Features that could be added:
    yml config generation.
    These files are needed by the GPRM to determine aliases in GPIR code, libraries used,
    number of threads/nodes. There is enough information at compilation time to generate these,
    and they are currently generated manually. 

    Other C++ language features, C++11 lambdas may be useful,
    while-loops, and possibly some STL support e.g. ability to use STD vector instead of pointers.






%The first page, abstract and table of contents are numbered using Roman numerals. From now on pages are numbered
%using Arabic numerals. Therefore, immediately after the first call to $\backslash$chapter we need the call
%$\backslash$pagenumbering$\{$arabic$\}$ and this should be called once only in the document. 

%The first Chapter should then be on page 1. You are allowed 50 pages for a 30 credit project and 35 pages for a 
%20 credit report. This includes everything up to but excluding the appendices and bibliograph, i.e. this is a limit on
%the body of the report.

%You are not allowed to alter text size (it is currently 11pt) neither are you allowed to alter the margins.

%Note that in this example, and some of the others, you need to execute the following commands the first time you process the files.
%Multiple calls to pdflatex are required to resolve references to labels and citations. The file bib.bib is the bibliography file.



%\begin{verbatim}

%            > pdflatex example0
%            > bibtex example0
%            > pdflatex example0
%            > pdflatex example0

%\end{verbatim}


%\section{First Section in Chapter}
%The quick brown fox jumped over the lazy dog.
%The quick brown fox jumped over the lazy dog \cite{DIMACS}.
%The quick brown fox jumped over the lazy dog.

%\subsection{A subsection}
%The quick brown fox jumped over the lazy dog.

%The quick brown fox \cite{fahle} jumped over the lazy dog.

%\chapter{The Fox and Dog}
%The quick brown fox jumped over the lazy dog.

%\section{The Fox Jumps Over}
%The quick brown fox jumped over the lazy dog.


%\vspace{-7mm}
%\begin{figure}
%\centering
%\includegraphics[height=9.2cm,width=13.2cm]{uroboros.pdf}
%\vspace{-30mm}
%\caption{An alternative hierarchy of the algorithms.}
%\label{uroborus}
%\end{figure}

%The quick brown fox jumped over the lazy dog.
%The quick brown fox jumped over \cite{ckt} the lazy dog.
%The quick brown fox jumped over the lazy dog.

%\section{The Lazy Dog}
%The quick brown fox jumped over the lazy dog.

%The quick brown fox jumped over the lazy dog.

%%%%%%%%%%%%%%%%
%              %
%  APPENDICES  %
%              %
%%%%%%%%%%%%%%%%
\begin{appendices}


%\chapter{Generating Random Graphs}
%\label{sec:randomGraph}
%We generate Erd\'{o}s-R\"{e}nyi random graphs $G(n,p)$ where $n$ is the number of vertices and
%each edge is included in the graph with probability $p$ independent from every other edge. It produces
%a random graph in DIMACS format with vertices numbered 1 to $n$ inclusive. It can be run from the command line as follows to produce 
%a clq file
\end{appendices}

%%%%%%%%%%%%%%%%%%%%
%   BIBLIOGRAPHY   %
%%%%%%%%%%%%%%%%%%%%

\bibliographystyle{plain}
\bibliography{bib}

\end{document}
