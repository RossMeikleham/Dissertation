\chapter{Implementation}


\section{Language Choice}

The compiler will be implemented using the Haskell programming language.

Functional languages are suited for building compilers.
This is due to the traversal of trees which occurs often during complication.
Functional languages typically have features such as pattern matching and tail-end recursive optimisation 
which make them suitable for traversing trees.

The Glasgow Haskell Compiler is available for most platforms, most importantly for Windows, OSX and Linux 
on x86 architectures. This allows the compiler to be portable across these platforms provided the libraries 
used to build the compiler are portable.

Haskell has algebaric datatypes, these makes Abstract Syntax Trees
simpler to implement. For example the AST for a very simple grammer
in Haskell can be represented as follows:

\begin{lstlisting}[style=myHaskell]

    data Expression =
          Add Expression Expression
        | Negate Expression 
        | Const Integer

\end{lstlisting}

The equivalent in an Imperative/OOP language (in this case Java)
would be the following:

\begin{lstlisting}[style=myJava]

abstract Class Expression {}

class Add extends Expression {
    public Expression left;
    public Expression right;
    public Add(Expression l, Expression r) {
        left = l;
        right = r;    
    } 
}

class Negate extends Expression {
    public Expression expr;
    public Negate(Expression e) {
        expr = e;    
    }
} 

class Const extends Expression {
    public int value;
    public Const(int v) {
        value = v;
    }
}       

\end{lstlisting}

The Haskell version is much clearer on the structure of the tree, and takes much
less code to implement. 

It is also to easily extend Haskell datatypes to include
custom annotations. For example storing the source code position for parts of expressions may
be useful for reporting errors to the user.

Due to the pure function nature of Haskell, parts of complilation can easily be performed in parallel.
For example during type checking, each GPC function can be type checked in parallel and this can even be subdivided 
further into blocks within functions. For this project parallisation is not a concern, but in the future if compliation ever
needs to be faster then this option is always available.

Haskell also has powerful libraries for parsing source code such as Parsec which is a parser combinator
library. Parsec allows Combinator parsers to be written in the Haskell language itself avoiding the complexity
of integration of different tools and languages. 
\cite{parsec}

\section{Tools and Testing}

\subsection{Cabal}

Cabal (Common Architecture for Building Applications and Libraries) is a system for building and
packaging Haskell libraries and programs.\cite{cabal}. Using this system to manage the
project library dependencies, and can automatically download and install missing dependencies,
build and install the compiler on the system, and run unit tests simply.

\subsection{Testing}

For Unit testing the HUnit library will be used, this can be integrated
with Cabal to easily run unit tests.

One form of testing used for this project is testing each individual component.
(e.g. The Parser, The TypeChecker) Each component in the compiler can easily
be uncoupled from one another due to the linear nature of compliation.

Another form of testing is writing GPC source files which should compile, and
GPC files which should fail compilation at a certain stage, the compiler
is then invoked during testing on all of these files to check whether
all the source files which should compile do infact compile with no errors,
and all the source files which should raise an error do not compile.

An upside to this method is that testing this way is flexible in that
they aren't coupled with the implementation internals of the compiler.
The only way that these test would need to be changed was if the design
of the language itself would need to be changed.

The downsides to this method is that without manually checking the errors raised
by the tests that should fail the source may generate an error which is unrelated
to the error that is being tested. This is why some testing of each internal
component is done alongside this method.

\subsection{Code Coverage}

The Haskell HPC (Haskell Program Coverage) library allows for recording
code coverage over different modules during testing. This can be integrated with
cabal unit testing to automatically generate these results. The usefulness of this
allows for checking which sections of code still need to be tested and assists
in writing further unit tests.


Splitting up compilation into multiple stages:
    
    Parsing/Lexing into AST

         |

     Type and Scope Checking the AST

         |

     Strip annotations and types from the AST

        |

     Evaluator which steps through the code and generates a 
     GPIR AST

        |

      Translate GPIR AST into GPIR source code

\section{Parser/Lexer}
\section{Type and Scope checking}
\section{Evaluation}
\section{GPIR Code Generation}
