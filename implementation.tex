\chapter{Implementation}


\section{Language Choice}

The compiler will be implemented using the Haskell programming language.

Functional languages are suited for building compilers.
This is due to the traversal of trees which occurs often during complication.
Functional languages typically have features such as pattern matching and tail-end recursive optimisation 
which make them suitable for traversing trees.

The Glasgow Haskell Compiler is available for most platforms, most importantly for Windows, OSX and Linux 
on x86 architectures. This allows the compiler to be portable across these platforms provided the libraries 
used to build the compiler are portable.

Haskell has algebaric datatypes, these make it easy to represent Abstract Syntax Trees in a 
readable form, and express recursive relations.

Also due to the pure function nature of Haskell, parts of complilation can easily be performed in parallel.
For example during type checking, each GPC function can be type checked in parallel and this can even be subdivided 
further into blocks within functions. For this project parallisation is not a concern, but in the future if compliation ever
needs to be faster then this option is always available.

\begin{lstlisting}[style=myHaskell]

    data Expression =
          Add Expression Expression
        | Sub Expression Expression
        | Const Integer

\end{lstlisting}



Haskell also has powerful libraries for parsing source code such as Parsec which is a parser combinator
library. Parsec allows Combinator parsers to be written in the Haskell language itself avoiding the complexity
of integration of different tools and languages. 
\cite{parsec}

\section{Tools and Build Process}

\subsection{Cabal}
\subsection{Testing}
\subsection{Continuous Integration}
\subsection{Code Coverage}


Splitting up compilation into multiple stages:
    
    Parsing/Lexing into AST

         |

     Type and Scope Checking the AST

         |

     Strip annotations and types from the AST

        |

     Evaluator which steps through the code and generates a 
     GPIR AST

        |

      Translate GPIR AST into GPIR source code

\section{Parser/Lexer}
\section{Type and Scope checking}
\section{Evaluation}
\section{GPIR Code Generation}
