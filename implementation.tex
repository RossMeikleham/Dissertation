\chapter{Implementation}

The language to implement the compile will be Haskell.
This is because functional languages are suited for compilers due to the nature of using Trees and recurrsively 
traversing them in program compilation.
Many functional languages have features such as Algebraic Datatypes, and tail-end recursive optimisations
which lend themselves well to these.
Haskell also has powerful libraries for parsing source code such as Parsec which is a parser combinator
library eliminating the need for such as yylex. It has a very powerful type system

Splitting up compilation into multiple stages:
    
    Parsing/Lexing into AST

         |

     Type and Scope Checking the AST

         |

     Strip annotations and types from the AST

        |

     Interpreter which generates a GPIR AST

        |

      Translate GPIR AST into GPIR source code
